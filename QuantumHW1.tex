\documentclass{article}
\usepackage[utf8]{inputenc}

\title{Quantum HW1}
\author{Samuel Bellenchia}
\date{February 2019}

\begin{document}

\maketitle

\section*{Wave Function Push-ups}
%\begin{multicols}{2}
\subsection*{1a}

%quick int; \int_{-\infty}^{\infty}=[infint]

$1={\displaystyle \int_{-\infty}^{\infty} }|Nxe^{-\alpha x}|^2dx$\\

$=N^2{\displaystyle \int_{0}^{\infty} }x^2e^{-2\alpha x}dx$\\

Integration by parts;\\

$=N^2(-\frac{x^2e^{-2\alpha x}}{2\alpha}|_0^{\infty}+\int_{-\infty}^{0}\frac{xe^{-2\alpha x}}{2\alpha}dx) $ \\

$=N^2(-\frac{x^2e^{-2\alpha x}}{2\alpha}|_0^{\infty}-\frac{xe^{-2\alpha x}}{2\alpha^2}+\int_{0}^{\infty}\frac{e^{-2\alpha x}}{2\alpha}) $ \\

$=N^2(-\frac{x^2e^{-2\alpha x}}{2\alpha}|_0^{\infty}-\frac{xe^{-2\alpha x}}{2\alpha^2}|_0^{\infty}-frac{e^{-2\alpha x}}{4\alpha^3}|_0^{\infty}) $ \\

$=-\frac{N^2}{4\alpha^3}[\lim\limits_{x\rightarrow\infty}(\frac{2x^2}{e^{2\alpha x}}+\frac{2x^2}{e^{2\alpha x}}+\frac{2x}{e^{2\alpha x}}+\frac{1}{e^{2\alpha x}})-(\frac{0}{e^{2\alpha 0}}+\frac{0}{e^{-2\alpha 0}}+\frac{1}{e^{2\alpha 0}})]$\\

L'Hopitals Rule;\\

$=\frac{-N^2}{4\alpha^3}(-1)\Rightarrow N^2=4\alpha^3$\\

$\Leftrightarrow N=2\alpha^{3/2}$\\

\subsection*{1b}

Is at maximum when the derivative equals zero\\

$\frac{d\psi}{dx}=Ne^{-\alpha x}(1-\alpha x)=0$\\

$\Leftrightarrow 1-\alpha x=0\Rightarrow x=1/\alpha$\\

\subsection*{1c}

$<\hat{x}>=\int_{-\infty}^{\infty}\psi^*(x)x\psi(x)dx$\\

$=\int_{-\infty}^{\infty}N^2x^3e^{-2\alpha x}$\\

We use integration by parts to evluate this integral;\\

$=N(-\frac{x^3}{2\alpha e^{2\alpha x}}-\frac{3x^2}{4\alpha^2 e^{2\alpha x}}-\frac{6x}{8\alpha^3 e^{2\alpha x}}-\frac{6}{16\alpha^4 e^{2\alpha x}})|_0^\infty$\\

Using L'Hopitals Rule to evaluate the limits as $x\rightarrow\infty$\\

$=N[(-0-0-0-0)-(-0-0-0-\frac{6}{16\alpha^4})]$

Using our value $N=2\alpha^{3/2}$ we get \\

$<\hat{x}>=\frac{6N^2}{16\alpha^4}=\frac{3}{2\alpha}$\\

\subsection*{1d}

$<\hat{x}^2>=\int_{-\infty}^{\infty}x^2|\psi(x)|^2dx$\\

$=\int_{-\infty}^{\infty}N^2x^4e^{-2\alpha x}$\\

We use integration by parts to evluate this integral;\\

$=N(-\frac{x^4}{2\alpha e^{2\alpha x}}-\frac{4x^3}{4\alpha^2 e^{2\alpha x}}-\frac{12x^2}{8\alpha^3 e^{2\alpha x}}-\frac{24x}{16\alpha^4 e^{2\alpha x}}-\frac{24}{32\alpha^5 e^{2\alpha x}})|_0^\infty$\\

Using L'Hopitals Rule to evaluate the limits as $x\rightarrow\infty$\\

$=N[(-0-0-0-0)-(-0-0-0-\frac{24}{32\alpha^5})]$

Using our value $N=2\alpha^{3/2}$ we get \\

$<\hat{x}^2>=\frac{24N^2}{32\alpha^5}=\frac{3}{\alpha^2}$\\

\subsection*{1e}

$\sigma_x=<x^2>-<x>^2$\\

$=\frac{3}{\alpha^2}-(\frac{3}{2\alpha})^2$\\

$\Rightarrow\sigma_x^2=\frac{3}{4\alpha^2}$\\

\subsection*{1f}

$<\hat{p}>=\int_{-\infty}^{\infty}\psi^*(-i\hbar\frac{d\psi}{dx})dx$\\

%$=\int_{-\infty}^{\infty}-i\hbar$\\

$=\int_{-\infty}^{\infty}i\hbar N^2e^{-2\alpha x}(\alpha x^2-x)dx$\\

We repeat the process from our previous steps to evaluate this integral;\\

$<p>=i\hbar N^2(\frac{2\alpha}{8\alpha^3}-\frac{1}{4\alpha^2})=0$\\

\subsection*{1g}

$<\hat{p}^2>=\int_{-\infty}^{\infty}\psi^*(-i\hbar\frac{d}{dx})^2\psi dx$\\


$=\int_{-\infty}^{\infty}Nxe^{-\alpha x}(-i\hbar\frac{d}{dx}(i\hbar Ne^{-\alpha x}(\alpha x-1)))dx$\\

$=\int_{-\infty}^{\infty}N^2\hbar^2e^{-2\alpha x}(2\alpha x-\alpha^2x^2)dx$\\

$=N^2\hbar^2[2\alpha(\frac{-x}{2\alpha e^{2\alpha x}}-\frac{1}{4\alpha^2 e^{2\alpha x}})-\alpha^2(\frac{-x^2}{2\alpha e^{2\alpha x}}-\frac{2x}{4\alpha^2 e^{2\alpha x}}-\frac{2}{8\alpha^3 e^{2\alpha x}})]|_0^\infty$\\

$=(\frac{N^2\hbar^2}{2\alpha}-\frac{N^2\hbar^2\alpha^2}{4\alpha^3})=\frac{N^2\hbar^2}{4\alpha}$\\

We now plug in N to get our final result;\\

$\Rightarrow<\hat{p}^2>=\hbar^2\alpha^2$\\

\subsection*{1h}

$\sigma_p=<p^2>-<p>^2$\\

$=\hbar^2\alpha^2-0$\\

$\Rightarrow\sigma_p^2=\hbar^2\alpha^2$\\

\subsection*{1i}

To verify Heisenburg's uncertainty princliple, we take the product of our two results' square roots;\\

$\sigma_x\sigma_p=\sqrt{\frac{3}{4\alpha^2}}\sqrt{\hbar^2\alpha^2}=\frac{\sqrt{3}\hbar}{2}$\\

Thus, it does in fact agree with the uncertainty principle.\\
%
%
%
%
%
%
%
%
\section*{Momentum Space}
\subsection*{2a}

Using the Fourier Transform, we can change our basis so that we're in momentum space.

$\phi(p)=\frac{1}{\sqrt{2\pi\hbar}}{\displaystyle \int_{-\infty}^{\infty} } Nxe^{-\alpha x}e^{-ipx/\hbar}dx$\\

$\phi(p)=\frac{1}{\sqrt{2\pi\hbar}}{\displaystyle \int_{-\infty}^{\infty} }Nxe^{-x(\alpha+ip/\hbar)}dx$\\

$\phi(p)=\frac{1}{\sqrt{2\pi\hbar}}[\frac{-xe^{-x(\alpha+ip/\hbar)}}{(\alpha+ip/\hbar)}-\frac{e^{-x(\alpha+ip/\hbar)}}{(\alpha+ip/\hbar)^2}]|_0^\infty$\\

$\phi(p)=\frac{1}{\sqrt{2\pi\hbar}}(0-0-(-0-\frac{1}{(\alpha+ip/\hbar)^2}))$\\

$\Rightarrow \phi(p)=\frac{N}{\sqrt{2\pi\hbar}(\alpha+ip/\hbar)^2}$\\

\subsection*{2b}

To normalize this, we first bring our imaginary unit into the numerator;\\

$\phi(p)=(\frac{N}{\sqrt{2\pi\hbar}})^2\frac{1}{\alpha^2+2ip\alpha/\hbar-p^2/\hbar^2}\frac{\alpha^2-2ip\alpha/\hbar-p^2/\hbar^2}{\alpha^2-2ip\alpha/\hbar+p^2/\hbar^2}=\frac{N^2}{2\pi\hbar}\frac{\alpha^2-2ip\alpha/\hbar-p^2/\hbar^2}{\alpha^4+4p^2\alpha^2/\hbar^2+p^4/\hbar^4}$\\

It's easy to see that multiplying by the conjugate again simply yields;\\

$1=\frac{N^2}{2\pi\hbar}{\displaystyle \int_{-\infty}^{\infty} }\frac{\alpha^4+2p^2\alpha^2/\hbar^2+p^4/\hbar^4}{(\alpha^4+2p^2\alpha^2/\hbar^2+p^4/\hbar^4)^2}=\frac{N^2}{2\pi\hbar}{\displaystyle \int_{-\infty}^{\infty} }\frac{1}{\alpha^4+2p^2\alpha^2/\hbar^2+p^4/\hbar^4}$\\

We simplify this expression for a special purpose (stay tuned to find out why!);\\ 

$1=\frac{N^2}{2\pi\hbar}{\displaystyle \int_{-\infty}^{\infty} } \frac{1}{(p^2/\hbar^2+\alpha^2)^2}dp=\frac{N^2}{2\pi\hbar}{\displaystyle \int_{-\infty}^{\infty} } \frac{\hbar^4}{(p^2+\alpha^2\hbar^2)^2}dp=\frac{N^2}{2\pi\hbar}{\displaystyle \int_{-\infty}^{\infty} } \frac{\hbar^4}{(p-i\alpha\hbar)^2(p+i\alpha\hbar)^2}dp$\\
\linebreak
%We can manipulate this equation
%
%$1=\frac{N^2}{2\pi\hbar}{\displaystyle \int_{-\infty}^{\infty} %}\frac{1}{(p/\hbar-i\alpha)^2(p/\hbar+i\alpha)^2}$\\
Cauchy's Residue Theorem states that if we have a function $f(z)$ which is complex analytic on a domain $D$ enclosed by some contour $\gamma$, the enclosement integral is equal to $2\pi i$ times the sum of its Residues on $D$\\

${\displaystyle \oint_\gamma }f(z)dz=2\pi i\sum_{i=0}^{2} Res_{z=z_i}(f(z_i))$\\

In order to enclose both residues and also have our path along the real line, we chose $\gamma=\gamma_1+\gamma_2$ such that \\

${\displaystyle \oint_{\gamma_1}}={\displaystyle \int_{[-R,R]}}+{\displaystyle \int_{C_1}}$, and ${\displaystyle \oint_{\gamma_2}}={\displaystyle \int_{[-R,R]}}+{\displaystyle \int_{C_2}}$\\

Where $C_1$ and $C_2$ are the following parameterized curves;\\

$C_1=Re^{i\theta},0<\theta<\pi$ and $C_2=Re^{-i\theta},0<\theta<\pi$
%$1=2\pi i[\frac{-2}{(-2i\alpha)^3}+\frac{-2}{(2i\alpha)^3}]$\\

Notice the two curves have opposite orientations, and when adding the residues we negate the latter.\\

$2\pi i\sum Res=2\pi i(\frac{d}{dp}[\frac{1}{(p+i\alpha\hbar)^2}]_{p=i\alpha\hbar}-\frac{d}{dp}[\frac{1}{(p-i\alpha\hbar)^2}]_{p=-i\alpha\hbar})$\\

$=2\pi i[\frac{-2}{(-2i\alpha)^3}-\frac{-2}{(2i\alpha)^3}]=\frac{\pi}{\alpha^4\hbar^4}$\\

Let us quickly show that the two curve's go to zero. First, by the triangle inequality (we apply it twice, breaking the polynomial each time) we see;\\

$|p^4+2\alpha^2\hbar^2p^2+\alpha^4\hbar^4|\geq|p^4+2\alpha^2\hbar^2p^2|+|\alpha^4\hbar^4|$\\

$|p^4|+|2\alpha^2\hbar^2p^2|+|\alpha^4\hbar^4|\geq R^4+2R^2\alpha^2\hbar^2+\alpha^4\hbar^4$\\

Using this, we can say the maximum value our function takes on this curve is less than the integration limit;\\

${max}_{p\in C}|\phi(p)|\leq\frac{1}{R^4+2R^2\alpha^2\hbar^2+\alpha^4\hbar^4}$\\

Finally, we can show by the arc length estimate that the two contour integrals $C_1$ and $C_2$ go to zero as $R\rightarrow\infty$\\

$|{\displaystyle\int_{C}}\phi(p)dp|\leq length(C)({max}_{p\in C})\leq\frac{\pi R}{R^4+2R^2\alpha^2\hbar^2+\alpha^4\hbar^4}$\\

$\Rightarrow \lim\limits_{R\rightarrow\infty}|{\displaystyle\int_{C}}\phi(p)dp|\rightarrow0$\\

Taking this limit means we have $\lim\limits_{R\rightarrow\infty}2{\displaystyle \int_{[-R,R]}}\phi(p)dp=\frac{\pi}{\alpha^4\hbar^4}$

Finally, back to our original equation;\\

$1=\frac{N^2\hbar^4}{2\pi\hbar}\frac{\pi}{2\alpha^3\hbar^3}=\frac{N^2}{4\alpha^3}$\\

$\Rightarrow N^2=4\alpha^3\Longrightarrow N=2\alpha^{3/2}$

\subsection*{2c}

We seek to find the expectation value by relation $<p>={\displaystyle \int_{-\infty}^{\infty} }\phi^*(p) p \phi(p) dp$\\

For brevity, we will use the same method as above without the rigorous proof of our claims and their applicability.\\ 

$<p>=\frac{N^2}{2\pi\hbar}{\displaystyle \int_{-\infty}^{\infty} }\frac{p\hbar^4}{\alpha^4+2p^2\alpha^2/\hbar^2+p^4/\hbar^4}=\frac{N^2\hbar^3}{2\pi}{\displaystyle \int_{-\infty}^{\infty}}\frac{p}{(p+i\alpha\hbar)^2(p-i\alpha\hbar)^2}$\\

$=2\pi i\frac{1}{2}[\frac{d}{dp}[\frac{p}{(p+i\alpha\hbar)^2}]_{p=i\alpha\hbar}-\frac{d}{dp}[\frac{p}{(p-i\alpha\hbar)^2}]_{p=-i\alpha\hbar}]$\\

$=\pi i[\frac{-4\alpha^2\hbar^2+4\alpha^2\hbar^2}{16\alpha^4\hbar^4}-\frac{-4\alpha^2\hbar^2+4\alpha^2\hbar^2}{16\alpha^4\hbar^4}]=0$\\

Thus our expectation value for momentum is zero, just as we found before transforming our wave funtion to momentum space.\\

\section*{Complex Absorbing Potentials}
\subsection*{3a}

For the standard Schroedinginger equation, we have $\frac{d}{dt}{\displaystyle \int_{-\infty}^{\infty}}dx|\Psi(x,t)|^2=0$\\

We now solve the equation using the "Gobbler" $V(x)=V_0(x)-i\Gamma$, assume our solution is seperable; as in $\Psi=\psi(x)\phi(t)$\\

$i\hbar\frac{d}{dt}\psi\phi=-\frac{\hbar^2}{2m}\frac{d^2}{dx^2}\psi\phi+(V_0(x)-i\Gamma)\psi\phi$\\

We divide both sides by $\psi(x)\phi(t)$ and rearrage to arrive at the following;\\

$i\hbar\frac{\phi"}{\phi}+\frac{\hbar^2}{2m}\frac{\psi''}{\psi}=V_0(x)-i\Gamma$\\

Since $V_0(x)$ is a function purely of x, we know our time dependant portion of the equation is equal to the latter term, giving us the following Ordinary Differential Equation;\\

$-i\hbar\frac{\phi'(t)}{\phi(t)}=i\Gamma$\\
$\phi'(t)=-\frac{\Gamma}{\hbar}\phi(t)$\\

Integrating this equation gives us the following result;\\

$\phi(t)=Const. e^{-t\Gamma/\hbar}$\\

Let us write $P={\displaystyle \int_{-\infty}^{\infty}}dx|\Psi(x,t)|^2$\\

Which means $\frac{d}{dt}{\displaystyle \int_{-\infty}^{\infty}}dx|\Psi(x,t)|^2=\frac{d}{dt}e^{-2t\Gamma/\hbar}{\displaystyle \int_{-\infty}^{\infty}}|Const.|^2|\psi(x)|^2dx$\\

$\Rightarrow \frac{dP}{dt}=\frac{-2\Gamma}{\hbar}P$\\

\subsection*{3b}

Solving for $P(t)$ simply means evaluating the differential equation we formed above;\\

$\frac{dP}{dt}=\frac{-2\Gamma}{\hbar}P\Rightarrow {\displaystyle\int}\frac{dP}{P}={\displaystyle\int}\frac{-2i\Gamma}{\hbar}dt$\\

$ln|P|=-2t\Gamma/\hbar+C$ Exponentiating both sides, we get\\

$P=Const*e^{-2t\Gamma/\hbar}$\\

Finding the lifetime of the particle simply means finding the value of $t$ which yields a value of $P=\frac{1}{e}$\\
This value is $t=\frac{\hbar}{2\Gamma}$\\


\section*{Normlization and Orthogonality}
\subsection*{4a}
Given a wavefuntion $\phi(x)=af(x)+bg(x)$ where $f(x)$ and $g(x)$ are orthonormal, normalizing the wavefunction yeilds;\\

%Let us assume that each of the complex constants $a$ and $b$ can be represented in the folloing forms; \\

%$a=u+iv$ and $b=p+iq$

$1={\displaystyle \int_{-\infty}^\infty}\phi^*(x)\phi(x)dx$\\

$=(|a|^2<f|f>+|b|^2<g|g>+a^*b<f|g>+b^*a<g|f>)$\\

$1=|a|^2+|b|^2$\\

\subsection*{4b}

Now, we deal with orthogonal functions which are not normalized; such that their overlap is given by ${\displaystyle \int_{-\infty}^\infty}f^*gdx=S$

$1=(|a|^2<f|f>+|b|^2<g|g>+a^*b<f|g>+b^*a<g|f>)$\\

$=(|a|^2+|b|^2+(a^*b)S+(b^*a)S)$\\

$=|a|^2+|b|^2+2(Re(a)Re(b)+Im(a)Im(b))$

Note that the later terms are the real and imaginary parts of a and b, respectively.

\section*{Quantum Cascade Lasers}

Damn bro what the actual fuck is this. You gotta stop sleeping through class dog, you're obviously behind becuase of your drug addict mentality. Step up your game young bloodddd\\


\end{document}
